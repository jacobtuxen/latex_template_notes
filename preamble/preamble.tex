%TEMPLATE BY JACOB TUXEN
%----------------basic pakages----------------
\usepackage[utf8]{inputenc}                     % for special chareacter like "æøå"
\usepackage[english]{babel}                      % for proper namning of autogenerated stuff
\usepackage[left=4.5cm,right=4.5cm,top=3cm,bottom=3cm]{geometry}  % for page size and margin setting
\usepackage{float}
\usepackage{calc}
\usepackage{hyperref}
\hypersetup{
	pdftitle={Assignment},
	colorlinks=false, linkcolor=doc!90,
	bookmarksnumbered=true,
	bookmarksopen=true
}                          % page numbers and '\ref's become clickable
\usepackage{subfiles} 
\usepackage{cite}
\usepackage{enumitem}                           % for control of 'enumerate'numbering
\usepackage{listings}                           % for control of 'itemize spacing
\usepackage{pdfpages}
\usepackage{xcolor}
\usepackage[T1]{fontenc} % Use T1 font encoding
\usepackage{bookmark}
\usepackage{tikz}
\usepackage{titlesec}
\usepackage{algorithmic}

%-----------images manipulation----------------
\usepackage{subcaption}                        % for more pictures in one figure
\usepackage{graphicx}    
\usepackage{wrapfig}                           % Allows figures or tables to have text wrapped around them
\usepackage{lscape}
\usepackage{svg}                               % for the automated integration of SVG graphics
                                               %  for for image manipulation
\graphicspath{{/graphics}}
%--------------math packages------------------
\usepackage{amssymb}
\usepackage{amsmath}
    \numberwithin{equation}{section}
\usepackage{amsthm}
\usepackage{mathtools}

%----------------tables-------------------------
\usepackage{colortbl}                           % colour in tables
\usepackage{makecell}                           % for  common tabular layouts 
\usepackage{longtable}
\usepackage{array}
\usepackage{tabu}
\usepackage{tabularx}
\usepackage{dcolumn}
\usepackage{multirow}
\usepackage{booktabs}                            % Enhances the quality of tables

%----------------header-&-footer----------------
\usepackage{fancyhdr}
\usepackage{lastpage}

%--------------Python code style--------------
\lstdefinestyle{mystyle}{
    language=Python,
    basicstyle=\ttfamily,
    keywordstyle=[1]\color{blue}, % def
    keywordstyle=[2]\color{orange!60!black}, % return
    commentstyle=\color{green!50!black},
    stringstyle=\color{red},
    showstringspaces=false,
    numbers=left,
    numberstyle=\tiny,
    numbersep=5pt,
    tabsize=4,
    breaklines=true,
    breakatwhitespace=true,
    belowcaptionskip=0pt,
    morekeywords={return}, % Add 'return' to keywords
    keywords=[1]{def}, % Define 'def' as a keyword of type 1
    keywords=[2]{return} % Define 'return' as a keyword of type 2
}

% Set default style for listings
\lstset{style=mystyle}