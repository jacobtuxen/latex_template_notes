\definecolor{doc}{RGB}{110,0,0}
\definecolor{myg}{RGB}{56, 140, 70}
\definecolor{myb}{RGB}{45, 111, 177}
\definecolor{myr}{RGB}{199, 68, 64}
\definecolor{mytheorembg}{HTML}{F2F2F9}
\definecolor{mytheoremfr}{HTML}{00007B}
\definecolor{myalgobg}{HTML}{ffecc6}
\definecolor{myalgofr}{HTML}{f88b00}
\definecolor{mylenmabg}{HTML}{FFFAF8}
\definecolor{mylenmafr}{HTML}{983b0f}
\definecolor{mypropbg}{HTML}{f2fbfc}
\definecolor{mypropfr}{HTML}{191971}
\definecolor{myexamplebg}{HTML}{F2FBF8}
\definecolor{myexamplefr}{HTML}{88D6D1}
\definecolor{myexampleti}{HTML}{2A7F7F}
\definecolor{mydefinitbg}{HTML}{E5E5FF}
\definecolor{mydefinitfr}{HTML}{3F3FA3}
\definecolor{notesgreen}{RGB}{0,162,0}
\definecolor{myp}{RGB}{197, 92, 212}
\definecolor{mygr}{HTML}{2C3338}
\definecolor{myred}{RGB}{127,0,0}
\definecolor{myyellow}{RGB}{169,121,69}
\definecolor{myexercisebg}{HTML}{F2FBF8}
\definecolor{myexercisefg}{HTML}{88D6D1}
\usepackage{titletoc}
%%%%%%%%%%%%%%%%%%%%%%%%%%%%%%%%%
% PACKAGE IMPORTS
%%%%%%%%%%%%%%%%%%%%%%%%%%%%%%%%%


\usepackage{amsmath,amsfonts,amsthm,amssymb,mathtools}
\usepackage[varbb]{newpxmath}
\usepackage{xfrac}
\usepackage[makeroom]{cancel}
\usepackage{mathtools}
\usepackage{bookmark}
\usepackage{enumitem}
\usepackage{theoremref}
\usepackage[most,many,breakable]{tcolorbox}
\usepackage{xcolor}
\usepackage{varwidth}
\usepackage{varwidth}
\usepackage{etoolbox}
%\usepackage{authblk}
\usepackage{nameref}
\usepackage{multicol,array}
\usepackage{tikz-cd}
\usepackage[ruled,vlined,linesnumbered]{algorithm2e}
\usepackage{comment} % enables the use of multi-line comments (\ifx \fi) 
\usepackage{import}
\usepackage{xifthen}
\usepackage{pdfpages}
\usepackage{transparent}


\newcommand\mycommfont[1]{\footnotesize\ttfamily\textcolor{blue}{#1}}
\SetCommentSty{mycommfont}
\newcommand{\incfig}[1]{%
    \def\svgwidth{\columnwidth}
    \import{./figures/}{#1.pdf_tex}
}

\usepackage{tikzsymbols}
\renewcommand\qedsymbol{$\Laughey$}

% Define custom formatting for table of contents
\contentsmargin{0cm}

% Custom formatting for sections in the table of contents
\titlecontents{section}[3.7pc]
{\addvspace{30pt}%
    \begin{tikzpicture}[remember picture, overlay]%
        \draw[fill=doc!60,draw=doc!60] (-7,-.1) rectangle (-0.9,.5);%
        \pgftext[left,x=-4.5cm,y=0.2cm]{\color{white}\Large\sc\bfseries section\ \thecontentslabel};%
    \end{tikzpicture}\color{doc!60}\large\sc\bfseries}%
{}
{}
{\;\titlerule\;\large\sc\bfseries Page \thecontentspage
    \begin{tikzpicture}[remember picture, overlay]
        \draw[fill=doc!60,draw=doc!60] (2pt,0) rectangle (4,0.1pt);
    \end{tikzpicture}}%

% Custom formatting for subsections in the table of contents
\titlecontents{subsection}[4.7pc] % Increased left margin for right alignment
{\addvspace{10pt}%
    \begin{tikzpicture}[remember picture, overlay]%
        \pgftext[left,x=-3.5cm,y=0.2cm]{\color{white}\Large\sc\bfseries Subsection\ \thecontentslabel};%
    \end{tikzpicture}\color{doc!60}\sc\bfseries}%
{}
{}
{\;\titlerule\;\large\sc\bfseries Page \thecontentspage
    \begin{tikzpicture}[remember picture, overlay]
        \draw[fill=doc!60,draw=doc!60] (2pt, 0) rectangle (4,0.1pt);
    \end{tikzpicture}}%

% Custom formatting for subsubsections in the table of contents
\titlecontents{subsubsection}[5.7pc] % Increased left margin for right alignment
{\addvspace{10pt}%
    \begin{tikzpicture}[remember picture, overlay]%
        \pgftext[left,x=-3.5cm,y=0.2cm]{\color{white}\Large\sc\bfseries Subsubsection\ \thecontentslabel};%
    \end{tikzpicture}\color{doc!60}\sc\bfseries}%
{}
{}
{\;\titlerule\;\large\sc\bfseries Page \thecontentspage
    \begin{tikzpicture}[remember picture, overlay]
        \draw[fill=doc!60,draw=doc!60] (2pt, 0) rectangle (4,0.1pt);
    \end{tikzpicture}}%

\makeatletter
\renewcommand{\tableofcontents}{%
    \section*{%
      \vspace*{-20\p@}%
      \begin{tikzpicture}[remember picture, overlay]%
          \pgftext[right,x=15cm,y=0.2cm]{\color{doc!60}\Huge\sc\bfseries \contentsname};%
          \draw[fill=doc!60,draw=doc!60] (13,-.75) rectangle (20,1);%
          \clip (13,-.75) rectangle (20,1);
          \pgftext[right,x=15cm,y=0.2cm]{\color{white}\Huge\sc\bfseries \contentsname};%
      \end{tikzpicture}}%
    \@starttoc{toc}}
\makeatother

%% HEADER FOOTER
\rfoot{Page \thepage ~of \pageref{LastPage}} % page number out of pages
\lfoot{\today}

%%  SECTION STYLES
% Customize section headings with TikZ
% Custom command for section titles with TikZ

\titleformat{\section}
{\color{doc}\normalfont\Large\bfseries}
{\color{doc}\thesection}{1em}{}

  
\titleformat{\subsection}
{\color{doc}\normalfont\large\bfseries}
{\color{doc}\thesubsection}{1em}{}

\titleformat{\subsubsection}
{\color{doc}\normalfont\normalsize\bfseries}
{\color{doc}\thesubsubsection}{1em}{}

%================================
% THEOREM BOX
%================================

\tcbuselibrary{theorems,skins,hooks}
\newtcbtheorem[number within=section]{Theorem}{Theorem}
{%
	enhanced,
	breakable,
	colback = mytheorembg,
	frame hidden,
	boxrule = 0sp,
	borderline west = {2pt}{0pt}{mytheoremfr},
	sharp corners,
	detach title,
	before upper = \tcbtitle\par\smallskip,
	coltitle = mytheoremfr,
	fonttitle = \bfseries\sffamily,
	description font = \mdseries,
	separator sign none,
	segmentation style={solid, mytheoremfr},
}
{th}

\tcbuselibrary{theorems,skins,hooks}
\newtcbtheorem[number within=chapter]{theorem}{Theorem}
{%
	enhanced,
	breakable,
	colback = mytheorembg,
	frame hidden,
	boxrule = 0sp,
	borderline west = {2pt}{0pt}{mytheoremfr},
	sharp corners,
	detach title,
	before upper = \tcbtitle\par\smallskip,
	coltitle = mytheoremfr,
	fonttitle = \bfseries\sffamily,
	description font = \mdseries,
	separator sign none,
	segmentation style={solid, mytheoremfr},
}
{th}


\tcbuselibrary{theorems,skins,hooks}
\newtcolorbox{Theoremcon}
{%
	enhanced
	,breakable
	,colback = mytheorembg
	,frame hidden
	,boxrule = 0sp
	,borderline west = {2pt}{0pt}{mytheoremfr}
	,sharp corners
	,description font = \mdseries
	,separator sign none
}

%================================
% NOTE BOX
%================================

\usetikzlibrary{arrows,calc,shadows.blur}
\tcbuselibrary{skins}
\newtcolorbox{note}[1][]{%
	enhanced jigsaw,
	colback=gray!20!white,%
	colframe=gray!80!black,
	size=small,
	boxrule=1pt,
	title=\textbf{Note:-},
	halign title=flush center,
	coltitle=black,
	breakable,
	drop shadow=black!50!white,
	attach boxed title to top left={xshift=1cm,yshift=-\tcboxedtitleheight/2,yshifttext=-\tcboxedtitleheight/2},
	minipage boxed title=1.5cm,
	boxed title style={%
			colback=white,
			size=fbox,
			boxrule=1pt,
			boxsep=2pt,
			underlay={%
					\coordinate (dotA) at ($(interior.west) + (-0.5pt,0)$);
					\coordinate (dotB) at ($(interior.east) + (0.5pt,0)$);
					\begin{scope}
						\clip (interior.north west) rectangle ([xshift=3ex]interior.east);
						\filldraw [white, blur shadow={shadow opacity=60, shadow yshift=-.75ex}, rounded corners=2pt] (interior.north west) rectangle (interior.south east);
					\end{scope}
					\begin{scope}[gray!80!black]
						\fill (dotA) circle (2pt);
						\fill (dotB) circle (2pt);
					\end{scope}
				},
		},
	#1,
}

%================================
% DEFINITION BOX
%================================

\usepackage{tcolorbox}
\tcbuselibrary{theorems}

% Definition box with numbering within sections
\newtcbtheorem[number within=section]{Definition}{Definition}{enhanced,
  before skip=2mm,after skip=2mm, colback=red!5, colframe=red!80!black, boxrule=0.5mm,
  attach boxed title to top left={xshift=1cm, yshift*=1mm-\tcboxedtitleheight}, varwidth boxed title*=-3cm,
  boxed title style={frame code={
      \path[fill=tcbcolback]
      ([yshift=-1mm,xshift=-1mm]frame.north west)
      arc[start angle=0,end angle=180,radius=1mm]
      ([yshift=-1mm,xshift=1mm]frame.north east)
      arc[start angle=180,end angle=0,radius=1mm];
      \path[left color=tcbcolback!60!black,right color=tcbcolback!60!black,
      middle color=tcbcolback!80!black]
      ([xshift=-2mm]frame.north west) -- ([xshift=2mm]frame.north east)
      [rounded corners=1mm]-- ([xshift=1mm,yshift=-1mm]frame.north east)
      -- (frame.south east) -- (frame.south west)
      -- ([xshift=-1mm,yshift=-1mm]frame.north west)
      [sharp corners]-- cycle;
    }, interior engine=empty,},
  fonttitle=\bfseries, title={#2},#1}{def}

%================================
% EXAMPLE BOX
%================================

\newtcbtheorem[number within=section]{Example}{Example}
{%
	colback = myexamplebg
	,breakable
	,colframe = myexamplefr
	,coltitle = myexampleti
	,boxrule = 1pt
	,sharp corners
	,detach title
	,before upper=\tcbtitle\par\smallskip
	,fonttitle = \bfseries
	,description font = \mdseries
	,separator sign none
	,description delimiters parenthesis
}
{ex}

%================================
% ALGORITHM BOX
%================================

\newtcbtheorem[number within=section]{Algo}{Algorithm}
{%
	enhanced,
	breakable,
	colback = myalgobg,
	frame hidden,
	boxrule = 0sp,
	borderline west = {2pt}{0pt}{myalgofr},
	sharp corners,
	detach title,
	before upper = \tcbtitle\par\smallskip,
	coltitle = myalgofr,
	fonttitle = \bfseries\sffamily,
	description font = \mdseries,
	separator sign none,
	segmentation style={solid, myalgofr},
}
{al}

\tcbuselibrary{theorems,skins,hooks}
\newtcolorbox{Algocon}
{%
	enhanced
	,breakable
	,colback = myalgobg
	,frame hidden
	,boxrule = 0sp
	,borderline west = {2pt}{0pt}{myalgofr}
	,sharp corners
	,description font = \mdseries
	,separator sign none
}

\newcommand{\thm}[2]{\begin{Theorem}{#1}{}#2\end{Theorem}}
\newcommand{\algo}[2]{\begin{Algo}{#1}{}#2\end{Algo}}
\newcommand{\cor}[2]{\begin{Corollary}{#1}{}#2\end{Corollary}}
\newcommand{\mlenma}[2]{\begin{Lenma}{#1}{}#2\end{Lenma}}
\newcommand{\mprop}[2]{\begin{Prop}{#1}{}#2\end{Prop}}
\newcommand{\clm}[3]{\begin{claim}{#1}{#2}#3\end{claim}}
\newcommand{\wc}[2]{\begin{wconc}{#1}{}\setlength{\parindent}{1cm}#2\end{wconc}}
\newcommand{\thmcon}[1]{\begin{Theoremcon}{#1}\end{Theoremcon}}
\newcommand{\algcon}[1]{\begin{Algocon}{#1}\end{Algocon}}
\newcommand{\ex}[2]{\begin{Example}{#1}{}#2\end{Example}}
\newcommand{\dfn}[2]{\begin{Definition}[colbacktitle=red!75!black]{#1}{}#2\end{Definition}}
\newcommand{\dfnc}[2]{\begin{definition}[colbacktitle=red!75!black]{#1}{}#2\end{definition}}
\newcommand{\qs}[2]{\begin{question}{#1}{}#2\end{question}}
\newcommand{\pf}[2]{\begin{myproof}[#1]#2\end{myproof}}
\newcommand{\nt}[1]{\begin{note}#1\end{note}}